% 8. Develop a LaTeX script to demonstrate the presentation of Numbered theorems, definitions,
% corollaries, and lemmas in the document.

\documentclass[a4paper,12pt]{article} 
\usepackage{amsmath, amssymb, amsthm} % For mathematical symbols and theorems % Define theorem environments  
\newtheorem{theorem}{Theorem}[section] % Theorem with numbering by section 
\newtheorem{lemma}[theorem]{Lemma} % Lemma with same numbering as theorems 
\newtheorem{corollary}[theorem]{Corollary} % Corollary with same numbering as theorems 
\newtheorem{definition}[theorem]{Definition} % Definition with same numbering as theorems 
\begin{document} 
\title{Demonstrating Numbered Theorems, Definitions, Corollaries, and Lemmas} 
\author{M Shivakumar} % Replace with your name 
\date{\today} 
\maketitle 
\section*{Introduction} In this document, we will present various mathematical results, such as 
theorems, lemmas, definitions, and corollaries, all properly numbered. 
\section{Theorems, Definitions, Lemmas, and Corollaries}  
\subsection*{Theorem Example} \begin{theorem} 
If \( a \) and \( b \) are two real numbers, then their sum is commutative, i.e.,  
\[ a + b = b + a. \] 
\end{theorem}  
\subsection*{Lemma Example}  
\begin{lemma}  
Let \( a \) and \( b \) be real numbers. If \( a + b = 0 \), then \( b = -a \).  
\end{lemma} 
\subsection*{Corollary Example}  
\begin{corollary} 
If \( a + b = 0 \) and \( a = 2 \), then \( b = -2 \).  
\end{corollary} 
\subsection*{Definition Example} 
\begin{definition}  
A number is called \textit{even} if it is divisible by 2. In other words, a number \( n \) is even if 
there exists an integer \( k \) such that  
\[ 
n = 2k. 
\]  
\end{definition}  
\section{Conclusion}  
This document demonstrates how to properly number and present theorems, lemmas, 
definitions, and corollaries in LaTeX using the `amsthm` package. You can easily refer to these 
results within your document, and LaTeX will handle the numbering automatically. 
\end{document}